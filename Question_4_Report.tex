\section*{Analytical Memo (250 words)}
The dual choropleth maps reveal stark spatial disparities in Uganda’s education access. Bushenyi, with an Enrollment Pressure Ratio of 380\% and a Primary Completion Ratio of 36\%, exemplifies strain where enrollment far exceeds completion, a trend prevalent in rural regions. This suggests overcrowding or dropout issues, potentially due to inadequate educational infrastructure despite economic and infrastructure advantages. Urban districts like Kampala show moderate pressure and higher completion, reflecting better resources. A correlation of -0.6351 between pressure and completion ratios quantifies this strain, explaining approximately 40\% of the variation.

**Policy Implications**: High-pressure, low-completion areas like Bushenyi require targeted interventions—additional classrooms, teacher recruitment—despite its economic reliance on agriculture and tourism. A 150\% pressure threshold flags high-risk districts for funding. Low-pressure areas (e.g., Amudat ~27.63\%) need enrollment boosts, possibly via mobile schools. Resource allocation must balance these needs.

**Areas of Highest Need**: Bushenyi’s 380\% pressure and 36\% completion signal critical strain, possibly worsened by over-enrollment outpacing school capacity, despite its economic edge from bananas, tea, and tourism, and infrastructure like roads to Mbarara. Rural zones with low completion despite moderate pressure may face access barriers. Economically, Bushenyi outperforms less diversified districts, yet educational outcomes lag, suggesting a disconnect between economic infrastructure and education investment. Further analysis (e.g., literacy maps) could reveal socioeconomic factors. The dashboards’ interactivity (filters, tooltips) aids dynamic exploration, supporting data-driven decisions to address Uganda’s education gaps.